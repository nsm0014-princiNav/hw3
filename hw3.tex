\documentclass[12pt,letterpaper, onecolumn]{exam}
\usepackage{amsmath}
\usepackage{pdfpages}
\usepackage{amssymb}
\usepackage{graphicx}
\usepackage{setspace}
\usepackage{nicefrac}
\usepackage{hyperref}
\setcounter{MaxMatrixCols}{20}
\usepackage[lmargin=71pt, tmargin=1.2in]{geometry}  %For centering solution box
\lhead{Principles of Navigation}
\rhead{Noah Miller}
\thispagestyle{empty}   %For removing header/footer from page 1

\begin{document}

\begingroup
\centering
\LARGE Principles of Navigation\\
\LARGE Homework 3 \\[0.5em]
\large \today\\[0.5em]
\large Noah Miller\par
\large 903949330\par
\large MECH 6970\par
\endgroup
\pointsdroppedatright   %Self-explanatory
\printanswers
\renewcommand{\solution}{\noindent\textbf{Answer:}\enspace}   %Replace "Ans:" with starting keyword in solution box



\begin{questions}
    \question{How many muiltiplies and additions are needed for each of the following computations?}
    \begin{parts}
        \part{Composition of rotations via rotation matrices, $\mathbf{C}_2^1\mathbf{C}_1^0$}

        \solution{%
            If we understand the rotation matrices to be in their most fundamental form, we can determine the number of multiplications and additions need to solve for $\mathbf{C}^0_2$. We start by defining both $\mathbf{C}_2^1$ and $\mathbf{C}_1^0$ which is a set of dot products between the respective frames.

            \begin{equation}
                \begin{split}
                    \mathbf{C}_2^1 & =
                    \begin{bmatrix}
                        x_2 \cdot x_1 & y_2 \cdot x_1 & z_2 \cdot x_1 \\
                        x_2 \cdot y_1 & y_2 \cdot y_1 & z_2 \cdot y_1 \\
                        x_2 \cdot z_1 & y_2 \cdot z_1 & z_2 \cdot z_1 \\
                    \end{bmatrix}
                \end{split}
                \label{eq:1}
            \end{equation}

            \begin{equation}
                \begin{split}
                    \mathbf{C}_1^0 & =
                    \begin{bmatrix}
                        x_1 \cdot x_0 & y_1 \cdot x_0 & z_1 \cdot x_0 \\
                        x_1 \cdot y_0 & y_1 \cdot y_0 & z_1 \cdot y_0 \\
                        x_1 \cdot z_0 & y_1 \cdot z_0 & z_1 \cdot z_0 \\
                    \end{bmatrix}
                \end{split}
                \label{eq:2}
            \end{equation}

            Defining these two rotations matrices we see that there 9 multiplications for each, so we add those to our running tally.
            Next, we multiply these together, which is a combination of multiplication and addition.
            \begin{equation}
                \begin{split}
                    \mathbf{C}_2^1\mathbf{C}_1^0 & =
                    \begin{bmatrix}
                        x_2 \cdot x_1 & y_2 \cdot x_1 & z_2 \cdot x_1 \\
                        x_2 \cdot y_1 & y_2 \cdot y_1 & z_2 \cdot y_1 \\
                        x_2 \cdot z_1 & y_2 \cdot z_1 & z_2 \cdot z_1 \\
                    \end{bmatrix}
                    \begin{bmatrix}
                        x_1 \cdot x_0 & y_1 \cdot x_0 & z_1 \cdot x_0 \\
                        x_1 \cdot y_0 & y_1 \cdot y_0 & z_1 \cdot y_0 \\
                        x_1 \cdot z_0 & y_1 \cdot z_0 & z_1 \cdot z_0 \\
                    \end{bmatrix} \\
                    \mathbf{C}_2^1\mathbf{C}_1^0 (:,1) & =
                    \begin{bmatrix}
                        \mathbf{C}_2^1(1,1)\mathbf{C}_1^0(1,1) + \mathbf{C}_2^1(1,2)\mathbf{C}_1^0(2,1) + \mathbf{C}_2^1(1,3)\mathbf{C}_1^0(3,1) \\
                        \mathbf{C}_2^1(2,1)\mathbf{C}_1^0(1,1) + \mathbf{C}_2^1(2,2)\mathbf{C}_1^0(2,1) + \mathbf{C}_2^1(2,3)\mathbf{C}_1^0(3,1) \\
                        \mathbf{C}_2^1(3,1)\mathbf{C}_1^0(1,1) + \mathbf{C}_2^1(3,2)\mathbf{C}_1^0(2,1) + \mathbf{C}_2^1(3,3)\mathbf{C}_1^0(3,1) \\
                    \end{bmatrix}
                \end{split}
                \label{eq:3}
            \end{equation}

            For the first column of matrix multiplication, there are 9 multiplications and 6 additions. We can triple this to account for the other two columns to add to our running tally.

            Adding our running total together, for part \ref{part@1@1}, we see there is 45 multiplications and 18 additions.
        }

        \part{Composition of rotations via quaternions, $\bar{q}_1^{\;2} \otimes \bar{q}_0^{\;1}$}

        \solution{%
            Counting the number of multiplications and additions of the tensor product between two quaternions is done easily if we layout the multiplication in its fundamental form.
            \begin{equation}
                \begin{split}
                    \bar{q}_1^{\;2} \otimes \bar{q}_0^{\;1} & =
                    \begin{bmatrix}
                        q_1^{2}(1) & -q_1^{2}(2) & -q_1^{2}(3) & -q_1^{2}(4) \\
                        q_1^{2}(2) & q_1^{2}(1)  & -q_1^{2}(4) & q_1^{2}(3)  \\
                        q_1^{2}(3) & q_1^{2}(4)  & q_1^{2}(1)  & q_1^{2}(2)  \\
                        q_1^{2}(4) & -q_1^{2}(3) & -q_1^{2}(2) & q_1^{2}(1)  \\
                    \end{bmatrix}
                    \begin{bmatrix}
                        q_0^{1}(1) \\
                        q_0^{1}(2) \\
                        q_0^{1}(3) \\
                        q_0^{1}(4) \\
                    \end{bmatrix}\\
                    \bar{q}_1^{\;2} \otimes \bar{q}_0^{\;1} & =
                    \begin{bmatrix}
                        q_1^{2}(1)q_0^{1}(1) + -q_1^{2}(2)q_0^{1}(2) + -q_1^{2}(3)q_0^{1}(3) + -q_1^{2}(4)q_0^{1}(4) \\
                        q_1^{2}(2)q_0^{1}(1) + q_1^{2}(1)q_0^{1}(2) + -q_1^{2}(4)q_0^{1}(3) + q_1^{2}(3)q_0^{1}(4)   \\
                        q_1^{2}(3)q_0^{1}(1) + q_1^{2}(4)q_0^{1}(2) + q_1^{2}(1)q_0^{1}(3) + q_1^{2}(2)q_0^{1}(4)    \\
                        q_1^{2}(4)q_0^{1}(1) + -q_1^{2}(3)q_0^{1}(2) + -q_1^{2}(2)q_0^{1}(3) + q_1^{2}(1)q_0^{1}(4)  \\
                    \end{bmatrix}
                \end{split}
                \label{eq:4}
            \end{equation}

            From here, we see a total of 16 multiplications and 12 additions.
        }

        \part{Recoordinatization of a vector via rotation matrix, $\mathbf{C}_2^1\vec{r}^{\;1}$}

        \solution{%
            From part \ref{part@1@1} we know the formation of $\mathbf{C}_2^1$ contains 9 multiplications. So now we just have to count the number of multiplications and additions that occur when evaluating $\mathbf{C}_2^1\vec{r}^{\;1}$.

            \begin{equation}
                \begin{split}
                    \mathbf{C}_2^1\vec{r}^{\;1} & =
                    \begin{bmatrix}
                        x_2 \cdot x_1 & y_2 \cdot x_1 & z_2 \cdot x_1 \\
                        x_2 \cdot y_1 & y_2 \cdot y_1 & z_2 \cdot y_1 \\
                        x_2 \cdot z_1 & y_2 \cdot z_1 & z_2 \cdot z_1 \\
                    \end{bmatrix}
                    \begin{bmatrix}
                        r_x^1 \\
                        r_y^1 \\
                        r_z^1 \\
                    \end{bmatrix}\\
                    \mathbf{C}_2^1\vec{r}^{\;1} & =
                    \begin{bmatrix}
                        \mathbf{C}_2^1(1,1)r_x^1 + \mathbf{C}_2^1(1,2)r_y^1 + \mathbf{C}_2^1(1,3)r_z^1 \\
                        \mathbf{C}_2^1(2,1)r_x^1 + \mathbf{C}_2^1(2,2)r_y^1 + \mathbf{C}_2^1(2,3)r_z^1 \\
                        \mathbf{C}_2^1(3,1)r_x^1 + \mathbf{C}_2^1(3,2)r_y^1 + \mathbf{C}_2^1(3,3)r_z^1 \\
                    \end{bmatrix}\\
                \end{split}
                \label{eq:5}
            \end{equation}
            From Equation \ref{eq:5} we see a total of 6 additions and 18 multiplications.
        }

        \part{Recoordinatization of a vector via quaternion, $\bar{q}_1^{\;2} \otimes \breve{r}^1 \otimes \left(\bar{q}_1^{\;2}\right)^{-1}$}

        \solution{%
        To evaluate the recoordinatization of a vector with quaternions, $\bar{q}_1^{\;2} \otimes \breve{r}^1 \otimes \left(\bar{q}_1^{\;2}\right)^{-1}$, we should be able to double the amount of multiplications and additions seen from part \ref{part@1@2}. Doing this, we see that two tensor products require 32 multiplications and 24 additions.
        }
    \end{parts}
    \clearpage
    \question{Consider the three-link, planar robot shown below for which four coordinate frames have been assigned. Frame $\{0\}$ is fixed, frame $\{1\}$ rotates with angle $\theta_1$ relative to Frame $\{0\}$, frame $\{2\}$ rotates with $\theta_2$ relative to frame $\{1\}$, and frame $\{3\}$ translates with distance $d_3$ relative to frame $\{2\}$.

        \begin{figure*}[!h]
            \centering
            \includegraphics[width=0.5\linewidth]{Q2ps.png}
        \end{figure*}

        The rotation matrices and displacements between frames are shown below.

        \centering
        $\mathbf{C}_1^0 =
            \begin{bmatrix}
                cos(\theta_1) & -sin(\theta_1) & 0 \\
                sin(\theta_1) & cos(\theta_1)  & 0 \\
                0             & 0              & 1 \\
            \end{bmatrix}
        $,
        $\vec{r}^{\;0}_{01} =
            \begin{bmatrix}
                a_1cos(\theta_1) \\
                a_1sin(\theta_1) \\
                0                \\
            \end{bmatrix}
        $,
        $\mathbf{C}_1^0 =
            \begin{bmatrix}
                cos(\theta_2) & 0  & -sin(\theta_2) \\
                sin(\theta_2) & 0  & cos(\theta_2)  \\
                0             & -1 & 0              \\
            \end{bmatrix}
        $

        \qquad \qquad \qquad \qquad \quad
        $\vec{r}^{\;1}_{12} =
            \begin{bmatrix}
                0 \\
                0 \\
                0 \\
            \end{bmatrix}$
        $\mathbf{C}_2^3 =
            \begin{bmatrix}
                1 & 0 & 0  \\
                0 & 0 & -1 \\
                0 & 1 & 0  \\
            \end{bmatrix}
        $,\quad
        $\vec{r}^{\;2}_{23} =
            \begin{bmatrix}
                0   \\
                0   \\
                d_3 \\
            \end{bmatrix}
        $
    }
    \begin{parts}
        \part{Determine the rotation matrix $\mathbf{C}_3^0$.}

        \part{Determine the translation vector $\vec{r}^{\;0}_{03}$}

        \part{Determine the following angular velocities as skew-symmetric matrices $\tilde{\omega}$ and vectors $\vec{\omega}$. Note $theta_1$,$\theta_2$, and $d_3$ can vary with time.}
        \begin{subparts}
            \subpart{$\tilde{\omega}^0_{01}$, $\vec{\omega}^0_{01}$}

            \subpart{$\tilde{\omega}^1_{12}$, $\vec{\omega}^1_{12}$}

            \subpart{$\tilde{\omega}^2_{23}$, $\vec{\omega}^2_{23}$}

            \subpart{$\tilde{\omega}^0_{03}$, $\vec{\omega}^0_{03}$}
        \end{subparts}

    \end{parts}
    \clearpage
    \question{Consider the time-varying coordinate transformation matrix $\mathbf{C}^{n}_{b}$ given below that describes the orientation of the body frame as it rotates with respect to the navigation frame.
        \[\mathbf{C}^n_b =
            \begin{bmatrix}
                cos(t)  & sin(t)sin(t^2)  & sin(t)cos(t^2) \\
                0       & cos(t^2)        & -sin(t^2)      \\
                -sin(t) & cos(t) sin(t^2) & cos(t)cos(t^2) \\
            \end{bmatrix} \]
    }
    \begin{parts}
        \part{Compute expressions for $\psi$,$\theta$, and $\phi$ based on fixed-axis definition of roll, pitch, yaw assuming a 1,2,3 series of rotations (roll,pitch,yaw).}

        \solution{%
            From class, we determined the Euler angles to be
            \begin{equation}
                \begin{split}
                    \phi & = tan^{-1}\left(\frac{-C_{23}}{C_{33}}\right) \\
                    \theta & = sin^{-1}(-C_{13})\\
                    \psi & = tan^{-1}\left(\frac{-C_{12}}{C_{11}}\right)\\
                \end{split}
                \label{eulerAngles}
            \end{equation}
        }

        \part{Use MATLAB to plot $\psi$,$\theta$, and $\phi$ as a function of time.}

        The simulation was run for 10 seconds with a sampling rate of 100 Hertz.

        \begin{figure}[!h]
            \centering
            \includegraphics[width=0.8\linewidth]{Euler_Angles.png}
            \caption{Roll, Pitch, and Yaw angles plotted as a function of time.}
            \label{fig:euler}
        \end{figure}

    \end{parts}
    \clearpage
    \question{Consider the time-varying coordinate transformation matrix $\mathbf{C}^n_b$ give below that describe the orientation of the body as it rotates with respect to the navigation frame.
        \[\mathbf{C}^n_b =
            \begin{bmatrix}
                cos(t)  & sin(t)sin(t^2)  & sin(t)cos(t^2) \\
                0       & cos(t^2)        & -sin(t^2)      \\
                -sin(t) & cos(t) sin(t^2) & cos(t)cos(t^2) \\
            \end{bmatrix} \]
    }
    \begin{parts}
        \part{Determine the analytic form of the time-derivative of $\mathbf{C}^n_b$ (i.e. $\dot{\mathbf{C}}^n_b = \frac{d\mathbf{C}^n_b}{dt}$) via a term-by-term differentiation.}

        We can use the \textit{Symbolic Math Toolbox} to differentiate the matrix for us very easily. Using \textit{diff} in MATLAB produces
        \[ \dot{\mathbf{C}}^n_b = \left[\begin{array}{ccc} -\sin\left(t\right) & \sin\left(t^2\right)\cos\left(t\right)+2t\cos\left(t^2\right)\sin\left(t\right) & \cos\left(t^2\right)\cos\left(t\right)-2t\sin\left(t^2\right)\sin\left(t\right)\\ 0 & -2t\sin\left(t^2\right) & -2t\cos\left(t^2\right)\\ -\cos\left(t\right) & 2t\cos\left(t^2\right)\cos\left(t\right)-\sin\left(t^2\right)\sin\left(t\right) & -\cos\left(t^2\right)\sin\left(t\right)-2t\sin\left(t^2\right)\cos\left(t\right) \end{array}\right]\]



        \part{Develop MATLAB functions which accept $t$ (i.e time) as a numerical input and return $\mathbf{C}^n_b$ and $\dot{\mathbf{C}}^n_b$, respectively, as numerical outputs.}

        \solution{%
            The created function is appended at the end of this submission.
        }

        \part{Using the $\mathbf{C}^n_b$ and $\dot{\mathbf{C}}^n_b$ functions from above, compute the angular velocity vector $\vec{\omega}^{\;n}_{nb}$ at time $t = 0$ seconds. (HINT: You might want to compute $\tilde{\omega}^{n}_{nb}$)}

        \solution{%
            For the following parts we can solve for $\vec{\omega}^{\;n}_{nb}$ using the following equation.
            \[\dot{\mathbf{C}}^n_b(\mathbf{C}^n_b)^{-1} = \tilde{\omega}^{\;n}_{nb} \]
            If we understand that $\tilde{\omega}^{\;n}_{nb}$ is just the skew-symmetric form of $\vec{\omega}^{\;n}_{nb}$, we can go between forms quite easily with Equation \ref{eq:10}.
            \begin{equation}
                \begin{split}
                    \tilde{\omega}^{\;n}_{nb} & =
                    \begin{bmatrix}
                        0         & -\omega_z & \omega_y  \\
                        \omega_z  & 0         & -\omega_x \\
                        -\omega_y & \omega_x  & 0         \\
                    \end{bmatrix}\\
                    \vec{\omega}^{\;n}_{nb} & =
                    \begin{bmatrix}
                        \omega_x \\
                        \omega_y \\
                        \omega_z \\
                    \end{bmatrix}\\
                \end{split}
                \label{eq:10}
            \end{equation}

            Using the aforementioned equations, we find that $\vec{\omega}^{\;n}_{nb}$ at time $t = 0$ seconds is
            \[\vec{\omega}^{\;n}_{nb}\vert_{(t = 0)}  =
                \begin{bmatrix}
                    0 \\
                    1 \\
                    0 \\
                \end{bmatrix} \]
        }
        \begin{subparts}
            \subpart{What is the magnitude (i.e. $\dot{\theta}$,angular speed) of the angular velocity?}

            \solution{%
                We can find the magnitude of $\vec{\omega}^{\;n}_{nb}\vert_{(t = 0)}$ by using \textit{norm} in MATLAB.
                \begin{equation}
                    \begin{split}
                        norm(\vec{\omega}^{\;n}_{nb}\vert_{(t = 0)}) & = 1
                    \end{split}
                    \label{eq:11}
                \end{equation}
            }

            \subpart{About what unit vector ($\vec{k}^{\;n}_{nb}$) has the instantaneous rotation occurred?}

            \solution{%
                We can find the unit vector, $\vec{k}^{\;n}_{nb}$, by first finding $\theta$. This can be done by using $\mathbf{C}^n_b$ evaluated at time $t = 0$ seconds.

                \begin{equation}
                    \begin{split}
                        \theta & = \cos^{-1}\left(\frac{trace(\mathbf{C}^n_b) - 1}{2} \right)\\
                        \theta & = 0^o \\
                    \end{split}
                    \label{eq:12}
                \end{equation}

                Then, using $\theta$ we can solve for the unit vector that the body is rotating about.
                \begin{equation}
                    \begin{split}
                        \vec{k}^{\;n}_{nb} & =
                        \begin{bmatrix}
                            \mathbf{C}^n_b(3,2) - \mathbf{C}^n_b(2,3) \\
                            \mathbf{C}^n_b(1,3) - \mathbf{C}^n_b(3,1) \\
                            \mathbf{C}^n_b(2,1) - \mathbf{C}^n_b(1,2) \\
                        \end{bmatrix}\frac{1}{2\sin\theta}
                    \end{split}
                    \label{eq:13}
                \end{equation}

                When doing this for time $t = 0$ seconds, we see that $\theta$ is 0, thus giving us an undefined unit vector. This is to be expected because the body did not have any initial angular velocity.
            }
        \end{subparts}

        \part{Using the $\mathbf{C}^n_b$ and $\dot{\mathbf{C}}^n_b$ functions from above, compute the angular velocity vector $\vec{\omega}^{\;n}_{nb}$ at time $t = 0.5$ seconds.}
        \begin{subparts}
            \subpart{What is the magnitude (i.e. $\dot{\theta}$,angular speed) of the angular velocity?}

            \subpart{About what unit vector ($\vec{k}^{\;n}_{nb}$) has the instantaneous rotation occurred?}
        \end{subparts}
        \part{Using the $\mathbf{C}^n_b$ and $\dot{\mathbf{C}}^n_b$ functions from above, compute the angular velocity vector $\vec{\omega}^{\;n}_{nb}$ at time $t = 1$ seconds.}
        \begin{subparts}
            \subpart{What is the magnitude (i.e. $\dot{\theta}$,angular speed) of the angular velocity?}

            \subpart{About what unit vector ($\vec{k}^{\;n}_{nb}$) has the instantaneous rotation occurred?}
        \end{subparts}

        \part{In practice, direct measurement of the angular velocity vector $\vec{\omega}^{\;n}_{nb}$ can prove challenging, so a finite-difference approach may be taken given two sequential orientations represented by $\mathbf{C}^n_b(t)$ and $\mathbf{C}^n_b(t + \Delta t)$ a small time $\Delta t$ apart. Consider the approximate value of the angular velocity vector $\vec{\omega}^{\;n}_{nb}$ derived by using the finite difference
            \[\dot{\mathbf{C}}^n_b(t) \approx \frac{\mathbf{C}^n_b(t + \Delta t) - \mathbf{C}^n_b(t)}{\Delta t} \]
            at times $t = 0$,$0.5$, and $1$ second. Compare the "analytic" values for $\dot{\theta}$ and $\vec{k}^{\;n}_{nb}$ (found in parts \ref{part@4@2}, \ref{part@4@3}, and \ref{part@4@4}) with your approximations from the finite difference using $\Delta t = 0.1$ seconds. How large are the errors?}
    \end{parts}
    \clearpage
    \question{Given the geodetic coordinate of the peak of Mt. Everest as Latitude ($L_b$) $27^o\;59'\;16"$ N, Longitude ($\lambda_b$) $86^o\;56'\;40"$ E, and height ($h_b$) 8850 meters (derived by GPS in 1999):}
    \begin{parts}
        \part{Develop a MATLAB function \[\text{function}\;\; r^e_{eb} = \text{llh2xyz}(L_b,\lambda_b,h_b) \]
        to convert from geodetic curvilinear Latitude, Longitude, and height to ECEF rectangular $x$,$y$, and $z$ coordinates (Please use SI units). Attach a printout of your function.}
        \begin{subparts}
            \subpart{Test your llh2xyz function using coordinates of the peak of Mt. Everest. What is $\vec{r}^{\;e}_{eb}$?}
        \end{subparts}

        \part{Develop a MATLAB function \[\text{function}\;\;\left[L_b,\lambda_b,h_b\right]  = \text{xyz2llh}(r^e_{eb}) \]
            to convert from ECEF rectangular $x$,$y$, and $z$ coordinates to geodetic curvilinear Latitude, Longitude, and height (Please use SI units). Attach a printout of your function.

            HINT: This should be an interactive transformation (i.e. not closed form).}

        \part{What is the acceleration due to gravity at the ellipsoid (i.e. at the ellipsoid $h_b = 0$. HINT: This should only be a function of Latitude -- see attached Pages)?}

        \part{What is the magnitude of the centrifugal acceleration ($-\tilde{\omega}^{e}_{ie}\,\tilde{\omega}^{e}_{ie}\,\vec{r}^{\;e_{eb}}$) at the ellipsoid and at the peak?}

        \part{What is the magnitude of the gravitational attraction at the ellipsoid and at the peak? HINT: See \textbf{attached pages} to compute $\vec{\gamma}^{\;e}_{ib} = \vec{\gamma}^{\;i}_{eb} \vert_{\vec{r}^{\;i}_{ib} = \vec{r}^{\;e}_{eb}}$}
    \end{parts}
\end{questions}
\includepdf[pages={1}]{Q4fcn.pdf}
\includepdf[pages={5-9}]{Hw_3.pdf}
\end{document}